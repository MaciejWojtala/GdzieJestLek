\documentclass[10pt, a4paper]{article}
\usepackage[T1]{fontenc}
\usepackage[polish]{babel}
\usepackage[utf8]{inputenc}
\usepackage{lmodern}
\usepackage{xcolor}
\selectlanguage{polish}
\author{Jakub Skrajny, Dawid Herman, Maciej Wojtala, Mateusz Błajda}
\title{Analiza wymagań I/II- "GdzieJestLek"}
\begin{document}
\maketitle
\section{Wprowadzenie}
 Dokument ma na celu przedstawić wymagania dotyczące działania aplikacji tworzonej w ramach projektu "GdzieJestLek". Wymagania dotyczą I fazy tworzenia projektu.
\section{Słownik}
\begin{itemize}
\item OBJKEY - para złożona z wartości typu 'Nazwa' i typu 'Substancja czynna'.
\item OBJ - zbiór wierszy, których pola 'Nazwa' i 'Substancja czynna' tworzą identyczny OBJKEY.
\item poprawne słowo - ciąg znaków, których dziesiętny kod ASCII należy do przedziału [32-122] oraz nie należy do żadnego z następujących przedziałów [33-39], [59], [63-64], [91-96].
\end{itemize}
Uwagi:\newline
Słowa 'rekord' oraz 'wiersz' są używane zamiennie. \newline
Niektóre definicje będą podawane na bieżąco.
\section{Ogólny opis}
\subsection{Co w I fazie}
Celem pierwszej fazy projektu "GdzieJestLek" jest stworzenie aplikacji webowej. Aplikacja będzie umożliwiała wyszukiwanie tekstu w bazie danych oraz wyświetlała wynik wyszukiwania.
\subsection{Baza danych}
Baza danych zostanie stworzona na podstawie dokumentu pt.
'załącznik do obwieszczenia' (później nazywany DOC). Obwieszczeniem, o którym mowa jest 'Obwieszczenie Ministra Zdrowia z dnia 18 lutego 2020 roku w sprawie wykazu refundowanych leków, środków spożywczych specjalnego przeznaczenia żywieniowego oraz wyrobów medycznych na 1 marca 2020 roku'. Wspomniany dokument zawiera informacje o lekach refundowaych, ich cenach, opakowaniach, substancjach czynnych, chorobach leczonych z ich pomocą itp..
\subsection{Interfejs}
Po włączeniu aplikacji, w przeglądarce pojawią się trzy elementy:\newline
\begin{itemize}
\item pole do wpisywania tekstu (nazywane później 'IN')
\item pole do wyświetlania wyniku - (nazywane później 'OUT') 
\item przycisk - (nazywane później 'SBT')
\end{itemize}
Jeżeli elementy, które powinny sie wyświetlić w OUT, nie będą się mieściły, to pod OUT pojawią się dwa nowe elementy. \newline
\begin{itemize}
\item przycisk ze strzałką w prawo (nazywany później R)
\item przycisk ze strzałką w lewo (nazywany później L)
\end{itemize}
\subsection{Działanie aplikacji}
Dwoma głównymi funkcjonalnościami, które zauważa użytkownik są wyszukiwanie oraz wyświetlanie. Po wciśnięciu SBT, zostanie podjęta próba wyszukania tekstu wpisanego do IN w bazie danych.
Następnie po przetworzeniu danych, wynik zostanie wypisany do OUT. Jeżeli wynik wyszukiwania okaże się zbyt duży, to pojawią się R oraz L.
Za pomocą nich będzie można zmienić zawartość OUT na dalsze/wcześniejsze części wyniku wyszukiwania.
\section{Wymagania Funkcjonalne}
\subsection{Baza danych}
Baza danych będzie zawierała jedną tabelę (Leki) o następujących kolumnach: \newline
\begin{itemize}
\item Nazwa
\item Substancja czynna
\item Postać
\item Dawka
\item Zawartość opakowania
\item Kod EAN lub inny kod odpowiadający kodowi EAN
\item Poziom odpłatności
\item Zakres wskazań objętych refundacją
\item Wysokość dopłaty świadczeniobiorcy
\end{itemize}
Wartośći w komórkach będą odpowiadały informacjom zawartym w DOC.\newline
W bazie danych nie będą występowały polskie znaki diakrytczne. Zamiast litery 'ą' należy używać 'a' itp. \newline
Znaki ';', (procent) oraz (Mikro) zostaną zamienione odpowiednio na '.', 'p', 'u'.
\subsection{Wyszukiwanie}
Wyszukiwanie rozpoczyna się po wciśnięciu przycisku SBT.
Aplikacja wyszukuje tekst będący zawartością IN (później nazywany TEXT) w momencie kliknięcia. \newline
Aplikacja nie wyszukuje słowa innego niż 'poprawne słowo' (wyjaśnienie w słowniku).\newline
Wynikiem wyszukiwania są wszystkie OBJ (wyjaśnienie w słowniku), które zawierają taką komórkę, że TEXT jest infiksem zawartości komórki.
\subsection{Wyświetlanie}
Po zakończeniu wyszukiwania, wynik wyświetli się w następujący sposób. \newline
Rekordy znajdujące się w wyniku zostaną wyświetlane w postaci tabeli. Kolumny wyświetlonej tabeli będą identyczne jak te w bazie danych. Kolejność wyświetlanych wierszy będzie leksykograficznie rosnąca przy czym porządkować będziemy po kolejnych wartościach kolumn od lewej do prawej. \newline
Wyniki wyszukiwania będą stronicowane, aby przyspieszyć czas przedstawiania ich w przeglądarce. W pol OUT zmieści się maksymalnie 100 wierszy. Na kolejnych stronach, które będzie można przeglądać przy pomocy przycisków L i R, pokazywane będzie 100(lub tyle co jeszcze pozostało) wierszy mniejsze/większe leksykograficznie.\newline 
W ramach jednego OBJ z wierszy zawartych w danej porcji do wyświetlenia, będzie dokonane złączenie pewnych komórek w jedną. Efekt złączenia w aplikacji będzie taki sam jak efekt złączenia komórek w skutek następującego procesu:\newline
1. Zaczynamy od kolumny nr 3,bo wartości w kolumnach 1 oraz 2 są takie same dla każdego rekordu w OBJ \newline
2. W aktualnej kolumnie łączymy komórki o tych samych wartościach, jeśli komórki z kolumny o numerze niższym o 1 już zostały połączone. Powtarzamy ten krok dla kolejnych kolumn.
\newline
\section{Wymagania niefunkcjonalne}
\subsection{Środowisko pracy aplikacji}
Aplikacja może działać poprawnie w dowolnej przeglądarce internetowej. Zapewniamy jej całkowicie poprawne działanie w następujących przeglądarkach:
\begin{itemize}
\item Google Chrome od wersji 29.0
\item Mozilla Firefoz od wersji 28.0
\item Safari od wersji 9.0
\item Opera od wersji 17.0
\end{itemize}
\subsection{Ograniczenia prawne}
Projekt nie zajmuje się kwestami prawnymi. W szczególności aplikacja w żaden sposób nie gwarantuje bezpieczeństwa wprowadzonych do niej danych medycznych.
\subsection{Wydajność}
Dołożone zostaną wszelkie starania, by szybkość reagowania aplikacji była zadowalająca dla każdego użytkownika. Nawet przy szesnastokrotnie większej bazie danych, zakładamy, że wyszukiwania i ładowania będą pomijalnie szybkie (czas trwania poniżej pół sekundy).
\end{document}
